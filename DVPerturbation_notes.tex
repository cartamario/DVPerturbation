\documentclass[12pt]{article}

%\usepackage[a4paper, bindingoffset=0.2inc,%
%            left=0.7in, right=0.7in, top=0.7in, bottom=1in,%
%            footskip=.25in]{geometry}
\usepackage{amsmath}
\usepackage{graphicx}
\usepackage{hyperref}
\usepackage[latin1]{inputenc}
\usepackage{verbatim}

\title{DVPerturbation Notes}
\author{Irene Virdis, Mario Carta}

\begin{document}
\maketitle
\section{Compulsory Modules}
The following modules are needed in order to correctly run all the command listed in the section below.

\begin{verbatim}
import numpy as np
from DVPerturbation import *
import os
import math
\end{verbatim}

\section{Class MeshDeform}
\subsection{Method ExtractSurface}
This method has been written to extract the coordinates of a surface, given its name as input; the expected file is an SU2 format.
In the lines below an example has been reported: we want to extract the blade surface "airfoil" from the mesh file "mesh.su2"; the python script for the istance of the class and the correct use of this method should contains the following commads:
\begin{verbatim}
object = MeshDeform(mesh_name='mesh.su2')
object.ExtractSurface(surface_name='airfoil')
\end{verbatim}

The method will automatically save into the folder a file renamed \textit{$blade_ points.txt$}, which contains: 
\begin{verbatim}
\begin{table}[ccc]
column 0       & column 1      & column 2    \\
index of point & X coordinate  & Y coordinate \\
\end{table}
\end{verbatim}






\end{document}

